\documentclass[t,aspectratio=169, xcolor={table}]{beamer}
\usepackage{listings}
\usepackage{hyperref}
\title{Long vectors for WebAssembly}
\author{Petr Penzin}
\institute{Intel Corporation}
\date{\today}
\begin{document}
\begin{frame}
  \titlepage
\end{frame}
\begin{frame}
\frametitle{Background}
  \begin{itemize}
  \item A number of discussions in the context of Wasm SIMD proposal regarding operations longer than 128-bit\footnotemark[1]\footnotemark[2]
  \item Existing runtime solutions
    \begin{itemize}
    \item \href{https://github.com/google/highway}{Highway}
    \item \href{https://docs.microsoft.com/en-us/dotnet/api/system.numerics.vector}{System.Numerics.Vector} in .NET
    \end{itemize}
  \end{itemize}
  \footnotetext[1]{
    Github issue \href{https://github.com/WebAssembly/simd/issues/29}{\#29}, and a version of this deck has been presented on November 6, 2019
  }
  \footnotetext[2]{
    \href{https://github.com/WebAssembly/simd/issues/210}{\#210}, \href{https://github.com/WebAssembly/simd/issues/212}{\#212}
  }
\end{frame}
\begin{frame}
\frametitle{Scope}
  This proposal is about:

  \begin{itemize}
  \item Length-agnostic vector operations which bridge the gap between hardware SIMD implementations
  \end{itemize}

  Out of scope:

  \begin{itemize}
  \item Supporting vector instruction sets, such as SVE (though it can in the future)
  \item Changing current Wasm SIMD proposal
  \end{itemize}
\end{frame}
\begin{frame}
\frametitle{Design Goals}
  \begin{itemize}
  \item Same Wasm binary to run all platforms
  \item Unambiguous instruction selection
  \item Easy transition from Wasm SIMD instruction set
  \end{itemize}
\end{frame}
\begin{frame}
\frametitle{Alternatives}
  Longer fixed-width SIMD WebAssembly ISA
  \begin{itemize}
  \item Not universally supported in hardware
  \item Goes against WebAssembly's design goal of representing the common set of operations between hardware platforms
  \item Cross platform code generation is challenging
  \end{itemize}
\end{frame}
\begin{frame}
\frametitle{Proposal}
  We propose length-agnostic variants of operations already present in Wasm \textit{simd128} proposal
  \begin{itemize}
  \item Loads and stores work with consecutive memory locations, like \textit{simd128} loads and stores
  \item Maximum vector length is set to match the hardware by runtime at startup
  \end{itemize}
\end{frame}
%\begin{frame}
%\frametitle{Vector length}
%  For a vector operation length can be:
%  \begin{enumerate}
%  \item Variable - number of elements processed determined when operation executes
%  \item Compile time constant - set when compiler runs, for example as in native SIMD compilation
%  \item Run time constant - set when runtime starts, constant for individual operations
%  \end{enumerate}
%
%  This proposal falls into run time constant category, which leaves room for compile-time vectorization algorithms, but still allows variety of SIMD instruction sets to be invoked at runtime.
%\end{frame}
\begin{frame}
\frametitle{Types and instructions}
New types and instructions
  \begin{itemize}
  \item $vec.<type>$ -- separate vector types for different lane types, size defaults to maximum supported by hardware
    \begin{itemize}
    \item $i8$, $i16$, $i32$, $i64$ -- integer
    \item $f32$, $f64$ -- floating point
    \end{itemize}
  \item $vec.<type>.length$ -- get number of elements in corresponding vector type
  \end{itemize}
\end{frame}
\begin{frame}
\frametitle{Types and instructions}
Instructions extending existing operations in WebAssembly SIMD proposal
  \begin{itemize}
  \item $vec.<type>.<op>$ -- same lane-wise operation as in \textit{simd128} $<op>$, applied to vector of $vec.<type>.length$
  \item[] For example, $vec.f32.mul$ is identical to $f32x4.mul$ on a 4-lane vector, $vec.i32.add$ to $i32x4.add$ , and so on
  \end{itemize}
\end{frame}
\begin{frame}[containsverbatim]
\frametitle{Example}
Vector addition, $c = a + b$, $sz$ is the size
\begin{lstlisting}
(block $loop
  (block $loop_top
    (br_if $loop (i32.lt (get_local $sz) (vec.f32.length)))
    vec.f32.load (get_local $a)
    vec.f32.load (get_local $b)
    vec.f32.add
    vec.f32.store (get_local $c)
    ;; Decrement $sz and increment $a, $b, $c
    (br $loop_top)
  )
)
(block $scalar_loop ;; Finish the remaining elements
\end{lstlisting}
\end{frame}
\begin{frame}
\frametitle{Code generation}
  \begin{itemize}
  \item Identical to \textit{simd128} for platforms that support only 128 bit SIMD
  \item Straight-forward extension to longer vectors on supporting platforms
  \end{itemize}
\end{frame}
\begin{frame}
\frametitle{Comparison against current SIMD proposal}
  \begin{itemize}
  \item At 128-bit vector width operations are identical to current Wasm SIMD operations with sole exception of lane shuffle
  \item Transparent to developer and toolchain
  \end{itemize}
\end{frame}
\begin{frame}
\frametitle{Poll}
Support phase 0 proposal for long vectors?
\end{frame}
\begin{frame}
\frametitle{~}
\huge{Thank you}
\end{frame}
\begin{frame}
\frametitle{Backup: Pure vectors}
  Proposal can be extended to support pure vectors - with user-visible length, but that would be challenging to execute on existing hardware.

  It can be done by adding the following instruction:

  \begin{itemize}
  \item $vec.<type>.set\_length$ -- set number of elements in corresponding vector type
  \item[] Takes an unsigned argument, allowed use smaller number per runtime's view of the hardware
  \end{itemize}
\end{frame}
\begin{frame}[containsverbatim]
\frametitle{Example}
Vector addition, $c = a + b$, $sz$ is the size
\begin{lstlisting}
local $len i32
(block $loop
  (block $loop_top
    (br_if $loop (i32.eq (get_local $sz) (i32.const 0)))
    (set_local $len (vec.f32.set_length (get_local $sz)))
    vec.f32.load (get_local $a)
    vec.f32.load (get_local $b)
    vec.f32.add
    vec.f32.store (get_local $c)
    ;; Decrement $sz by $len; increment $a, $b, and $c by $len
    (br $loop_top)
  )
)
\end{lstlisting}
\end{frame}
\begin{frame}
\frametitle{Code generation}
  Advantages:
  \begin{itemize}
  \item Reduced Wasm instruction count
  \item Some alignment with SIMD instruction sets supporting masking
  \end{itemize}
  Disadvantages:
  \begin{itemize}
  \item High cost for SIMD instruction sets without masking
  \item Managing global state 
  \end{itemize}
  This can be seen as a future or experimental option, but it is not ready to be prototyped on widely available hardware.
\end{frame}
\end{document}

